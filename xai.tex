\documentclass[12pt,a4paper]{article}
\usepackage[utf8]{inputenc}
\usepackage[T2A]{fontenc}
\usepackage[ukrainian]{babel}
\usepackage{geometry}
\geometry{margin=1in}
\usepackage{booktabs}
\usepackage{longtable}
\usepackage{array}
\usepackage{multirow}
\usepackage{hyperref}
\usepackage{sectsty}
\usepackage{tocloft}
\usepackage{fancyhdr}
\usepackage{parskip}

\sectionfont{\centering\Large\bfseries}
\subsectionfont{\large\bfseries}

\hypersetup{
    colorlinks=true,
    linkcolor=blue,
    citecolor=blue,
    urlcolor=blue,
}

\pagestyle{fancy}
\fancyhf{}
\rhead{Дослідження xAI}
\lhead{30 грудня 2025 р.}
\cfoot{\thepage}

\title{Дослідження вендора великих мовних моделей: xAI}
\author{}
\date{30 грудня 2025 р.}

\begin{document}

\maketitle

\tableofcontents
\newpage

\section{Словник термінів}

\begin{itemize}
    \item \textbf{Трансформер} — архітектура нейромережі для обробки послідовностей, основа LLM.
    \item \textbf{RLHF (Reinforcement Learning from Human Feedback)} — навчання з підкріпленням на основі відгуків людей для покращення відповідей.
    \item \textbf{MoE (Mixture of Experts)} — архітектура, де модель активує лише частину параметрів для ефективності (наприклад, 2 експерти на токен у Grok-1).
    \item \textbf{Квантизація} — зменшення точності параметрів (наприклад, 8-бітна) для оптимізації інференсу.
    \item \textbf{Контекстне вікно} — максимальна довжина вхідних даних (наприклад, 8K токенів у Grok-1, до 2M у Grok-4.1 Fast).
    \item \textbf{Мультимодальність} — обробка тексту, зображень, аудіо тощо (Grok-4.1 підтримує voice agents).
    \item \textbf{Агентні інструменти} — функції для автономних дій, як пошук, код-виконання, інтеграція з X.
    \item \textbf{Токенізатор} — розбиття тексту на токени (SentencePiece у Grok-1 з 131K токенами).
    \item \textbf{Rotary Embeddings (RoPE)} — позиційне кодування для довгих контекстів.
    \item \textbf{API} — інтерфейс для розробників (наприклад, Grok Voice Agent API).
\end{itemize}

\section{Вступ: xAI як архітектор ШІ для розуміння Всесвіту}

Компанія xAI заснована в липні 2023 року Ілоном Маском з метою ``розуміння справжньої природи Всесвіту'' через ШІ. xAI інтегрує штучний інтелект з екосистемою X (колишній Twitter), Tesla та SpaceX, фокусуючись на даних реального часу, швидкості та агентних можливостях.

На відміну від Meta (open weights для Llama), xAI відкрила ваги лише для Grok-1, але пропонує потужні API для розробників. Компанія інвестує мільярди доларів, будує суперкомп'ютери (наприклад, Memphis Supercluster). 

Переваги підходу: інтеграція з X для пошуку в реальному часі, voice agents та освітні ініціативи (партнерство з Ель-Сальвадором для 1 млн школярів). Станом на грудень 2025 року випущено Grok-4.1 — лідер у LMSYS та аудіо-бенчмарках.

\section{Еволюція моделей xAI}

Хронологічний огляд моделей:

\begin{itemize}
    \item \textbf{Grok-1 (2023)} — базова MoE-модель з 314B параметрів, тренована на даних X, фокус на гуморі та відповідях у реальному часі.
    \item \textbf{Grok-1.5 (2024)} — покращення в кодуванні, математиці та довгому контексті.
    \item \textbf{Grok-2 (2024)} — збільшена ефективність, інтеграція з X для пошуку.
    \item \textbf{Grok-3 (лютий 2025)} — доступ для преміум-користувачів, фокус на reasoning та мультимодальності.
    \item \textbf{Grok-3.5 (2025)} — покращення в agentic tasks.
    \item \textbf{Grok-4 (липень 2025)} — революційний стрибок у reasoning.
    \item \textbf{Grok-4.1 (листопад 2025)} — швидка версія з 2M контекстом, voice agents, інтеграція з Tesla.
    \item \textbf{Grok-5} — очікується в Q1 2026, подвоєні параметри.
\end{itemize}

Еволюція: від базового чат-бота до мультимодальних агентів з пошуком у реальному часі (X/web), voice та інструментами.

\section{Карта моделей}

\begin{longtable}{|>{\centering\arraybackslash}p{3cm}|>{\centering\arraybackslash}p{3.5cm}|>{\centering\arraybackslash}p{2.5cm}|>{\centering\arraybackslash}p{2cm}|>{\centering\arraybackslash}p{2.5cm}|>{\centering\arraybackslash}p{2cm}|>{\centering\arraybackslash}p{3cm}|>{\centering\arraybackslash}p{3cm}|}
\hline
\textbf{Модель} & \textbf{Розмір (параметрів)} & \textbf{Тип/Архітектура} & \textbf{Дата релізу} & \textbf{Контекстне вікно} & \textbf{Токенізатор} & \textbf{Ліцензія/Доступ} & \textbf{Особливості} \\ \hline
\endhead

Grok-1 & 314B (MoE, 2 експерти/токен) & MoE, 64 шари & Листопад 2023 & 8K токенів & SentencePiece (131K токенів) & Apache 2.0 (відкриті ваги) & RoPE, 8-біт квантизація, JAX код \\ \hline
Grok-1.5 & Невідомо (покращена) & MoE & Березень 2024 & 128K & Невідомо & API-доступ & Покращений код/математика \\ \hline
Grok-2 & Невідомо & MoE & Серпень 2024 & 128K+ & Невідомо & API & Інтеграція з X \\ \hline
Grok-3 & Невідомо & MoE & Лютий 2025 & 512K+ & Невідомо & Преміум & Reasoning, мультимодальність \\ \hline
Grok-3.5 & Невідомо & MoE & 2025 & Невідомо & Невідомо & API & Agentic tasks \\ \hline
Grok-4 & Невідомо (значно більший) & MoE & Липень 2025 & 1M+ & Невідомо & API & Advanced reasoning \\ \hline
Grok-4.1 & Невідомо & MoE, RL для довгого контексту & Листопад 2025 & 2M токенів & Невідомо & API (\$0.2/1M вхід) & Voice agents, \#1 у LMSYS \\ \hline
Grok-4.1 Fast & Невідомо & Оптимізована & Листопад 2025 & 2M & Невідомо & Безкоштовно 2 тижні & Agent Tools API, 5x швидше \\ \hline

\caption{Порівняльна таблиця моделей xAI}
\end{longtable}

\section{Ключові персони}

\begin{itemize}
    \item \textbf{Ілон Маск} — засновник та CEO, візіонер, інтегрує xAI з Tesla/X.
    \item \textbf{Ігор Бабушкін} — головний інженер, екс-DeepMind.
    \item Команда: ~100 інженерів з Google, OpenAI, DeepMind (наприклад, Грег Янг, Тобі Поулінгер).
\end{itemize}

xAI активно приваблює таланти з конкурентів.

\section{Бенчмарки та порівняння}

\begin{itemize}
    \item Grok-4: 65 балів (другий після GPT-5), лідер LMSYS.
    \item Grok-4.1: \#1 на Big Bench Audio, 5x швидше OpenAI у voice.
    \item Перевершує конкурентів у MMLU, MATH, кодингу, agentic search.
    \item Перевага в pronunciation, accent, prosody.
\end{itemize}

\section{Інструменти та підходи}

\begin{itemize}
    \item \textbf{Репозиторії}: \url{https://github.com/xai-org/grok-1} (JAX код для Grok-1, Apache 2.0).
    \item \textbf{API}: Grok Voice Agent API (\$0.05/хв), підтримує десятки мов, інструменти (X/web search, code execution).
    \item \textbf{Agent Tools API}: автономний браузинг, пошук X, код-виконання (безкоштовно 2 тижні в листопаді 2025).
    \item \textbf{Інтеграції}: Tesla (voice в авто), X (real-time data), освіта.
    \item \textbf{Оптимізації}: Long-horizon RL, in-house voice stack.
\end{itemize}

\section{Висновки}

xAI демократизує ШІ через інтеграцію з реальними даними платформи X та агентні можливості, фокусуючись на швидкості та практичності. Еволюція: від відкритої Grok-1 до мультимодальних Grok-4.1 з voice та tools. 

Обмеження: не всі ваги відкриті, сильна залежність від екосистеми Ілона Маска. Компанія лідирує в voice/audio бенчмарках.

\section{Джерела}

\begin{enumerate}
    \item \url{https://x.ai/news/grok-4-1}
    \item \url{https://vertu.com/lifestyle/the-ai-model-race-reaches-singularity-speed}
    \item \url{https://doit.software/blog/grok-statistics}
    \item \url{https://www.nextbigfuture.com/2025/11/xai-releases-grok-4-1-and-it-tops-the-lmarena-leaderboard.html}
    \item \url{https://blog.promptlayer.com/grok-5-what-we-expect}
    \item \url{https://originality.ai/blog/grok-ai-statistics}
    \item \url{https://www.datastudios.org/post/grok-4-vs-previous-models-...}
    \item \url{https://smythos.com/developers/ai-models/whats-new-in-grok-4-release-facts-benchmarks-and-value}
    \item \url{https://magai.co/grok-4-release}
    \item \url{https://artsmart.ai/blog/grok-ai-statistics}
    \item GitHub: \url{https://github.com/xai-org/grok-1}
    \item Анонси та пости від @xai на платформі X (2024--2025 рр.).
\end{enumerate}

\end{document}
